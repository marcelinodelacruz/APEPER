% Options for packages loaded elsewhere
\PassOptionsToPackage{unicode}{hyperref}
\PassOptionsToPackage{hyphens}{url}
\PassOptionsToPackage{dvipsnames,svgnames,x11names}{xcolor}
%
\documentclass[
  letterpaper,
  DIV=11,
  numbers=noendperiod]{scrreprt}

\usepackage{amsmath,amssymb}
\usepackage{lmodern}
\usepackage{iftex}
\ifPDFTeX
  \usepackage[T1]{fontenc}
  \usepackage[utf8]{inputenc}
  \usepackage{textcomp} % provide euro and other symbols
\else % if luatex or xetex
  \usepackage{unicode-math}
  \defaultfontfeatures{Scale=MatchLowercase}
  \defaultfontfeatures[\rmfamily]{Ligatures=TeX,Scale=1}
\fi
% Use upquote if available, for straight quotes in verbatim environments
\IfFileExists{upquote.sty}{\usepackage{upquote}}{}
\IfFileExists{microtype.sty}{% use microtype if available
  \usepackage[]{microtype}
  \UseMicrotypeSet[protrusion]{basicmath} % disable protrusion for tt fonts
}{}
\makeatletter
\@ifundefined{KOMAClassName}{% if non-KOMA class
  \IfFileExists{parskip.sty}{%
    \usepackage{parskip}
  }{% else
    \setlength{\parindent}{0pt}
    \setlength{\parskip}{6pt plus 2pt minus 1pt}}
}{% if KOMA class
  \KOMAoptions{parskip=half}}
\makeatother
\usepackage{xcolor}
\setlength{\emergencystretch}{3em} % prevent overfull lines
\setcounter{secnumdepth}{5}
% Make \paragraph and \subparagraph free-standing
\ifx\paragraph\undefined\else
  \let\oldparagraph\paragraph
  \renewcommand{\paragraph}[1]{\oldparagraph{#1}\mbox{}}
\fi
\ifx\subparagraph\undefined\else
  \let\oldsubparagraph\subparagraph
  \renewcommand{\subparagraph}[1]{\oldsubparagraph{#1}\mbox{}}
\fi

\usepackage{color}
\usepackage{fancyvrb}
\newcommand{\VerbBar}{|}
\newcommand{\VERB}{\Verb[commandchars=\\\{\}]}
\DefineVerbatimEnvironment{Highlighting}{Verbatim}{commandchars=\\\{\}}
% Add ',fontsize=\small' for more characters per line
\usepackage{framed}
\definecolor{shadecolor}{RGB}{241,243,245}
\newenvironment{Shaded}{\begin{snugshade}}{\end{snugshade}}
\newcommand{\AlertTok}[1]{\textcolor[rgb]{0.68,0.00,0.00}{#1}}
\newcommand{\AnnotationTok}[1]{\textcolor[rgb]{0.37,0.37,0.37}{#1}}
\newcommand{\AttributeTok}[1]{\textcolor[rgb]{0.40,0.45,0.13}{#1}}
\newcommand{\BaseNTok}[1]{\textcolor[rgb]{0.68,0.00,0.00}{#1}}
\newcommand{\BuiltInTok}[1]{\textcolor[rgb]{0.00,0.23,0.31}{#1}}
\newcommand{\CharTok}[1]{\textcolor[rgb]{0.13,0.47,0.30}{#1}}
\newcommand{\CommentTok}[1]{\textcolor[rgb]{0.37,0.37,0.37}{#1}}
\newcommand{\CommentVarTok}[1]{\textcolor[rgb]{0.37,0.37,0.37}{\textit{#1}}}
\newcommand{\ConstantTok}[1]{\textcolor[rgb]{0.56,0.35,0.01}{#1}}
\newcommand{\ControlFlowTok}[1]{\textcolor[rgb]{0.00,0.23,0.31}{#1}}
\newcommand{\DataTypeTok}[1]{\textcolor[rgb]{0.68,0.00,0.00}{#1}}
\newcommand{\DecValTok}[1]{\textcolor[rgb]{0.68,0.00,0.00}{#1}}
\newcommand{\DocumentationTok}[1]{\textcolor[rgb]{0.37,0.37,0.37}{\textit{#1}}}
\newcommand{\ErrorTok}[1]{\textcolor[rgb]{0.68,0.00,0.00}{#1}}
\newcommand{\ExtensionTok}[1]{\textcolor[rgb]{0.00,0.23,0.31}{#1}}
\newcommand{\FloatTok}[1]{\textcolor[rgb]{0.68,0.00,0.00}{#1}}
\newcommand{\FunctionTok}[1]{\textcolor[rgb]{0.28,0.35,0.67}{#1}}
\newcommand{\ImportTok}[1]{\textcolor[rgb]{0.00,0.46,0.62}{#1}}
\newcommand{\InformationTok}[1]{\textcolor[rgb]{0.37,0.37,0.37}{#1}}
\newcommand{\KeywordTok}[1]{\textcolor[rgb]{0.00,0.23,0.31}{#1}}
\newcommand{\NormalTok}[1]{\textcolor[rgb]{0.00,0.23,0.31}{#1}}
\newcommand{\OperatorTok}[1]{\textcolor[rgb]{0.37,0.37,0.37}{#1}}
\newcommand{\OtherTok}[1]{\textcolor[rgb]{0.00,0.23,0.31}{#1}}
\newcommand{\PreprocessorTok}[1]{\textcolor[rgb]{0.68,0.00,0.00}{#1}}
\newcommand{\RegionMarkerTok}[1]{\textcolor[rgb]{0.00,0.23,0.31}{#1}}
\newcommand{\SpecialCharTok}[1]{\textcolor[rgb]{0.37,0.37,0.37}{#1}}
\newcommand{\SpecialStringTok}[1]{\textcolor[rgb]{0.13,0.47,0.30}{#1}}
\newcommand{\StringTok}[1]{\textcolor[rgb]{0.13,0.47,0.30}{#1}}
\newcommand{\VariableTok}[1]{\textcolor[rgb]{0.07,0.07,0.07}{#1}}
\newcommand{\VerbatimStringTok}[1]{\textcolor[rgb]{0.13,0.47,0.30}{#1}}
\newcommand{\WarningTok}[1]{\textcolor[rgb]{0.37,0.37,0.37}{\textit{#1}}}

\providecommand{\tightlist}{%
  \setlength{\itemsep}{0pt}\setlength{\parskip}{0pt}}\usepackage{longtable,booktabs,array}
\usepackage{calc} % for calculating minipage widths
% Correct order of tables after \paragraph or \subparagraph
\usepackage{etoolbox}
\makeatletter
\patchcmd\longtable{\par}{\if@noskipsec\mbox{}\fi\par}{}{}
\makeatother
% Allow footnotes in longtable head/foot
\IfFileExists{footnotehyper.sty}{\usepackage{footnotehyper}}{\usepackage{footnote}}
\makesavenoteenv{longtable}
\usepackage{graphicx}
\makeatletter
\def\maxwidth{\ifdim\Gin@nat@width>\linewidth\linewidth\else\Gin@nat@width\fi}
\def\maxheight{\ifdim\Gin@nat@height>\textheight\textheight\else\Gin@nat@height\fi}
\makeatother
% Scale images if necessary, so that they will not overflow the page
% margins by default, and it is still possible to overwrite the defaults
% using explicit options in \includegraphics[width, height, ...]{}
\setkeys{Gin}{width=\maxwidth,height=\maxheight,keepaspectratio}
% Set default figure placement to htbp
\makeatletter
\def\fps@figure{htbp}
\makeatother
\newlength{\cslhangindent}
\setlength{\cslhangindent}{1.5em}
\newlength{\csllabelwidth}
\setlength{\csllabelwidth}{3em}
\newlength{\cslentryspacingunit} % times entry-spacing
\setlength{\cslentryspacingunit}{\parskip}
\newenvironment{CSLReferences}[2] % #1 hanging-ident, #2 entry spacing
 {% don't indent paragraphs
  \setlength{\parindent}{0pt}
  % turn on hanging indent if param 1 is 1
  \ifodd #1
  \let\oldpar\par
  \def\par{\hangindent=\cslhangindent\oldpar}
  \fi
  % set entry spacing
  \setlength{\parskip}{#2\cslentryspacingunit}
 }%
 {}
\usepackage{calc}
\newcommand{\CSLBlock}[1]{#1\hfill\break}
\newcommand{\CSLLeftMargin}[1]{\parbox[t]{\csllabelwidth}{#1}}
\newcommand{\CSLRightInline}[1]{\parbox[t]{\linewidth - \csllabelwidth}{#1}\break}
\newcommand{\CSLIndent}[1]{\hspace{\cslhangindent}#1}

\KOMAoption{captions}{tableheading}
\makeatletter
\@ifpackageloaded{tcolorbox}{}{\usepackage[many]{tcolorbox}}
\@ifpackageloaded{fontawesome5}{}{\usepackage{fontawesome5}}
\definecolor{quarto-callout-color}{HTML}{909090}
\definecolor{quarto-callout-note-color}{HTML}{0758E5}
\definecolor{quarto-callout-important-color}{HTML}{CC1914}
\definecolor{quarto-callout-warning-color}{HTML}{EB9113}
\definecolor{quarto-callout-tip-color}{HTML}{00A047}
\definecolor{quarto-callout-caution-color}{HTML}{FC5300}
\definecolor{quarto-callout-color-frame}{HTML}{acacac}
\definecolor{quarto-callout-note-color-frame}{HTML}{4582ec}
\definecolor{quarto-callout-important-color-frame}{HTML}{d9534f}
\definecolor{quarto-callout-warning-color-frame}{HTML}{f0ad4e}
\definecolor{quarto-callout-tip-color-frame}{HTML}{02b875}
\definecolor{quarto-callout-caution-color-frame}{HTML}{fd7e14}
\makeatother
\makeatletter
\makeatother
\makeatletter
\@ifpackageloaded{bookmark}{}{\usepackage{bookmark}}
\makeatother
\makeatletter
\@ifpackageloaded{caption}{}{\usepackage{caption}}
\AtBeginDocument{%
\ifdefined\contentsname
  \renewcommand*\contentsname{Indice de contenidos}
\else
  \newcommand\contentsname{Indice de contenidos}
\fi
\ifdefined\listfigurename
  \renewcommand*\listfigurename{Listado de Figuras}
\else
  \newcommand\listfigurename{Listado de Figuras}
\fi
\ifdefined\listtablename
  \renewcommand*\listtablename{Listado de Tablas}
\else
  \newcommand\listtablename{Listado de Tablas}
\fi
\ifdefined\figurename
  \renewcommand*\figurename{Figura}
\else
  \newcommand\figurename{Figura}
\fi
\ifdefined\tablename
  \renewcommand*\tablename{Tabla}
\else
  \newcommand\tablename{Tabla}
\fi
}
\@ifpackageloaded{float}{}{\usepackage{float}}
\floatstyle{ruled}
\@ifundefined{c@chapter}{\newfloat{codelisting}{h}{lop}}{\newfloat{codelisting}{h}{lop}[chapter]}
\floatname{codelisting}{Listado}
\newcommand*\listoflistings{\listof{codelisting}{Listado de Listatdos}}
\makeatother
\makeatletter
\@ifpackageloaded{caption}{}{\usepackage{caption}}
\@ifpackageloaded{subcaption}{}{\usepackage{subcaption}}
\makeatother
\makeatletter
\@ifpackageloaded{tcolorbox}{}{\usepackage[many]{tcolorbox}}
\makeatother
\makeatletter
\@ifundefined{shadecolor}{\definecolor{shadecolor}{rgb}{.97, .97, .97}}
\makeatother
\makeatletter
\makeatother
\ifLuaTeX
  \usepackage{selnolig}  % disable illegal ligatures
\fi
\IfFileExists{bookmark.sty}{\usepackage{bookmark}}{\usepackage{hyperref}}
\IfFileExists{xurl.sty}{\usepackage{xurl}}{} % add URL line breaks if available
\urlstyle{same} % disable monospaced font for URLs
\hypersetup{
  pdftitle={Análisis de patrones de puntos ecológicos con R},
  pdfauthor={Marcelino de la Cruz Rot},
  colorlinks=true,
  linkcolor={blue},
  filecolor={Maroon},
  citecolor={Blue},
  urlcolor={Blue},
  pdfcreator={LaTeX via pandoc}}

\title{Análisis de patrones de puntos ecológicos con R}
\author{Marcelino de la Cruz Rot}
\date{8/3/2022}

\begin{document}
\maketitle
\ifdefined\Shaded\renewenvironment{Shaded}{\begin{tcolorbox}[borderline west={3pt}{0pt}{shadecolor}, enhanced, boxrule=0pt, interior hidden, frame hidden, breakable, sharp corners]}{\end{tcolorbox}}\fi

\renewcommand*\contentsname{Indice de contenidos}
{
\hypersetup{linkcolor=}
\setcounter{tocdepth}{2}
\tableofcontents
}
\bookmarksetup{startatroot}

\hypertarget{pruxf3logo}{%
\chapter*{Prólogo}\label{pruxf3logo}}
\addcontentsline{toc}{chapter}{Prólogo}

This is a Quarto book.

To learn more about Quarto books visit
\url{https://quarto.org/docs/books}.

\bookmarksetup{startatroot}

\hypertarget{resumen}{%
\chapter*{Resumen}\label{resumen}}
\addcontentsline{toc}{chapter}{Resumen}

In summary, this book has no content whatsoever.

\bookmarksetup{startatroot}

\hypertarget{introducciuxf3n}{%
\chapter{Introducción}\label{introducciuxf3n}}

\hypertarget{ejemplo-buxe1sico-de-anuxe1lisis-espacial}{%
\section{Ejemplo básico de análisis
espacial}\label{ejemplo-buxe1sico-de-anuxe1lisis-espacial}}

\texttt{dist}

See Knuth (1984) for additional discussion of literate programming.

\bookmarksetup{startatroot}

\hypertarget{codificaciuxf3n-de-patrones-de-puntos-en-r}{%
\chapter{Codificación de patrones de puntos en
R}\label{codificaciuxf3n-de-patrones-de-puntos-en-r}}

Como hemos visto en la introducción, todo patrón de puntos está definido
por las coordenadas de los puntos que lo integran más la ``ventana'' (el
polígono) que lo delimita.

\hypertarget{lectura-de-datos}{%
\section{Lectura de datos}\label{lectura-de-datos}}

\texttt{read.xlsx} Leer shapefiles Lectura de imagenes/transformación a
formato img

\begin{Shaded}
\begin{Highlighting}[]
\FunctionTok{library}\NormalTok{(xlsx)}
\NormalTok{datos}\OtherTok{\textless{}{-}} \FunctionTok{read.xlsx}\NormalTok{(}\StringTok{"datos.xlsx"}\NormalTok{, }\AttributeTok{sheetIndex=}\DecValTok{1}\NormalTok{)}
\FunctionTok{head}\NormalTok{(datos)}
\DocumentationTok{\#\# Poner aquí un trozo de la salida}
\end{Highlighting}
\end{Shaded}

\begin{tcolorbox}[enhanced jigsaw, colback=white, breakable, rightrule=.15mm, opacityback=0, arc=.35mm, toprule=.15mm, leftrule=.75mm, bottomrule=.15mm, colframe=quarto-callout-color-frame, left=2mm]

La generación de interpolaciones estadísticas está fuera del objetivo
del libro, pero por conveniencia se incluye aquí un ejemplo de la
herramienta \texttt{automap}

\begin{Shaded}
\begin{Highlighting}[]
\CommentTok{\# POner un ejemplo con automap y su conversión a im de spatstat.}
\DocumentationTok{\#\# Poner aquí un trozo de la salida}
\end{Highlighting}
\end{Shaded}

\end{tcolorbox}

\hypertarget{definiciuxf3n-de-la-ventana}{%
\section{Definición de la ventana}\label{definiciuxf3n-de-la-ventana}}

Lo más conveniente suele ser comenzar definiendo la ventana. En
\textbf{spatstat} lo hacemos con la función \texttt{owin}. Por ejemplo,
si nuestros puntos están recogidos dentro de un cuadrado que se extiende
entre las coordenadas relativas
\texttt{x=\ c(0,\ 100)\ ,\ y\ =\ c(0,\ 100)}, sólo deberemos indicarle a
\texttt{owin} la extensión de la ventana en el eje \(x\) y en el eje
\(y\).

\begin{Shaded}
\begin{Highlighting}[]
\FunctionTok{library}\NormalTok{(spatstat)}
\NormalTok{ventana }\OtherTok{\textless{}{-}} \FunctionTok{owin}\NormalTok{(}\AttributeTok{xrange =} \FunctionTok{c}\NormalTok{(}\DecValTok{0}\NormalTok{, }\DecValTok{100}\NormalTok{), }\AttributeTok{yrange=}\FunctionTok{c}\NormalTok{(}\DecValTok{0}\NormalTok{,}\DecValTok{100}\NormalTok{))}
\NormalTok{ventana}
\DocumentationTok{\#\# window: rectangle = [0, 100] x [0, 100] units}
\end{Highlighting}
\end{Shaded}

Si la ventana es estrictamente cuadrada, también la podemos definir
usando la función \texttt{square} e indicando el tamaño del lado de la
misma.

\begin{Shaded}
\begin{Highlighting}[]
\NormalTok{ventana }\OtherTok{\textless{}{-}} \FunctionTok{owin}\NormalTok{(}\FunctionTok{square}\NormalTok{(}\AttributeTok{r=}\DecValTok{100}\NormalTok{))}
\NormalTok{ventana}
\DocumentationTok{\#\# window: rectangle = [0, 100] x [0, 100] units}
\end{Highlighting}
\end{Shaded}

La ventana creada (Figura~\ref{fig-ventana}) la podemos visualizar con
\texttt{plot(ventana)}.

\begin{figure}

{\centering \includegraphics{./images/ventana.png}

}

\caption{\label{fig-ventana}Ventana creada con la función owin}

\end{figure}

En el caso de ventanas irregulares, las coordenadas de los puntos que
definen los vértices de la misma se proporcionan con el argumento
\texttt{poly} y deben ir en forna de lista (\texttt{list}), con dos
componentes denominados específicamente \(x\) e \(y\):

\begin{Shaded}
\begin{Highlighting}[]
\NormalTok{borde }\OtherTok{\textless{}{-}} \FunctionTok{list}\NormalTok{(}\AttributeTok{x=}\FunctionTok{c}\NormalTok{(}\FloatTok{0.2}\NormalTok{,}\FloatTok{0.3}\NormalTok{,}\FloatTok{0.5}\NormalTok{, }\FloatTok{0.8}\NormalTok{,}\FloatTok{0.6}\NormalTok{,}\FloatTok{0.3}\NormalTok{),}
          \AttributeTok{y=}\FunctionTok{c}\NormalTok{(}\FloatTok{0.1}\NormalTok{,}\FloatTok{0.1}\NormalTok{,}\FloatTok{0.2}\NormalTok{, }\FloatTok{0.5}\NormalTok{,}\FloatTok{0.7}\NormalTok{,}\FloatTok{0.3}\NormalTok{))}
\NormalTok{ventana2 }\OtherTok{\textless{}{-}} \FunctionTok{owin}\NormalTok{(}\AttributeTok{poly=}\NormalTok{ borde)}
\NormalTok{ventana2}
\DocumentationTok{\#\# window: polygonal boundary}
\DocumentationTok{\#\# enclosing rectangle: [0.2, 0.8] x [0.1, 0.7] units}
\end{Highlighting}
\end{Shaded}

\begin{tcolorbox}[enhanced jigsaw, left=2mm, titlerule=0mm, opacitybacktitle=0.6, leftrule=.75mm, colframe=quarto-callout-warning-color-frame, toptitle=1mm, colback=white, breakable, rightrule=.15mm, opacityback=0, colbacktitle=quarto-callout-warning-color!10!white, coltitle=black, bottomtitle=1mm, title=\textcolor{quarto-callout-warning-color}{\faExclamationTriangle}\hspace{0.5em}{Advertencia}, arc=.35mm, bottomrule=.15mm, toprule=.15mm]
Es importante que las coordenadas de los vértices estén ordenadas en
sentido antihorario.
\end{tcolorbox}

\hypertarget{creaciuxf3n-del-patruxf3n-de-puntos}{%
\section{Creación del patrón de
puntos}\label{creaciuxf3n-del-patruxf3n-de-puntos}}

\hypertarget{coordenadas-absolutas-o-coordenadas-relativas}{%
\section{Coordenadas absolutas o coordenadas
relativas}\label{coordenadas-absolutas-o-coordenadas-relativas}}

\texttt{affine.ppp} y \texttt{affine.owin}

\hypertarget{patrones-de-puntos-marcados}{%
\section{Patrones de puntos
marcados}\label{patrones-de-puntos-marcados}}

\hypertarget{marcas-discretas}{%
\subsection{Marcas discretas}\label{marcas-discretas}}

\hypertarget{marcas-continuas}{%
\subsection{Marcas continuas}\label{marcas-continuas}}

\hypertarget{marcas-muxfaltiples}{%
\subsection{Marcas múltiples}\label{marcas-muxfaltiples}}

\hypertarget{marcas-funcionales}{%
\subsection{Marcas funcionales}\label{marcas-funcionales}}

See Knuth (1984) for additional discussion of literate programming.

\bookmarksetup{startatroot}

\hypertarget{descripciuxf3n-de-patrones-de-puntos}{%
\chapter{Descripción de patrones de
puntos}\label{descripciuxf3n-de-patrones-de-puntos}}

\hypertarget{funciones-sumario}{%
\section{Funciones sumario}\label{funciones-sumario}}

\hypertarget{funciuxf3n-k-de-ripley}{%
\subsection{Función K de Ripley}\label{funciuxf3n-k-de-ripley}}

\hypertarget{funciuxf3n-de-correlaciuxf3n-de-par-y-o-ring}{%
\subsection{Función de correlación de par y
O-Ring}\label{funciuxf3n-de-correlaciuxf3n-de-par-y-o-ring}}

\hypertarget{otras-funciones-h-f-j-d-ver-las-de-wiegand-poarar-perfect-simulation}{%
\subsection{Otras funciones: H, F, J, D (ver las de Wiegand poarar
perfect
simulation)}\label{otras-funciones-h-f-j-d-ver-las-de-wiegand-poarar-perfect-simulation}}

\hypertarget{test-csr}{%
\section{Test CSR}\label{test-csr}}

\hypertarget{descripciuxf3n-de-la-intensidad-patrones-inhomoguxe9neos-y-agregaciuxf3n-virtual}{%
\section{Descripción de la intensidad: patrones inhomogéneos y
agregación
virtual}\label{descripciuxf3n-de-la-intensidad-patrones-inhomoguxe9neos-y-agregaciuxf3n-virtual}}

\hypertarget{dependencia-test-de-berman-igual-dejarlo-para-modelizaciuxf3n}{%
\section{Dependencia: test de Berman (igual dejarlo para
modelización?)}\label{dependencia-test-de-berman-igual-dejarlo-para-modelizaciuxf3n}}

See Knuth (1984) for additional discussion of literate programming.

\bookmarksetup{startatroot}

\hypertarget{modelizaciuxf3n-de-patrones-de-puntos}{%
\chapter{Modelización de patrones de
puntos}\label{modelizaciuxf3n-de-patrones-de-puntos}}

\hypertarget{modelos-poisson}{%
\section{Modelos Poisson}\label{modelos-poisson}}

\hypertarget{modelos-adhoc}{%
\section{Modelos adhoc}\label{modelos-adhoc}}

\hypertarget{modelos-de-gibbs}{%
\section{Modelos de Gibbs}\label{modelos-de-gibbs}}

\hypertarget{etc}{%
\section{Etc}\label{etc}}

This is a book created from markdown and executable code.

See Knuth (1984) for additional discussion of literate programming.

\bookmarksetup{startatroot}

\hypertarget{modelos-nulos-y-test-de-hipuxf3tesis}{%
\chapter{Modelos nulos y test de
hipótesis}\label{modelos-nulos-y-test-de-hipuxf3tesis}}

\hypertarget{simulaciones-y-envueltas}{%
\section{Simulaciones y envueltas}\label{simulaciones-y-envueltas}}

\hypertarget{gof-mad-test-etc}{%
\section{GoF, MAD test etc}\label{gof-mad-test-etc}}

\hypertarget{modelos-paramuxe9tricos-ppm-etc}{%
\section{Modelos paramétricos (ppm
etc)}\label{modelos-paramuxe9tricos-ppm-etc}}

\hypertarget{modelos-no-paramuxe9tricos}{%
\section{Modelos no paramétricos}\label{modelos-no-paramuxe9tricos}}

\hypertarget{perfect-simulation}{%
\section{Perfect simulation}\label{perfect-simulation}}

See Knuth (1984) for additional discussion of literate programming.

\bookmarksetup{startatroot}

\hypertarget{anuxe1lisis-de-patrones-de-puntos-replicados}{%
\chapter{Análisis de patrones de puntos
replicados}\label{anuxe1lisis-de-patrones-de-puntos-replicados}}

\hypertarget{test-no-paramuxe9tricos}{%
\section{Test no paramétricos}\label{test-no-paramuxe9tricos}}

\hypertarget{ajuste-de-modelos-para-patrones-replicados}{%
\section{Ajuste de modelos para patrones
replicados}\label{ajuste-de-modelos-para-patrones-replicados}}

See Knuth (1984) for additional discussion of literate programming.

\bookmarksetup{startatroot}

\hypertarget{patrones-con-marcas-continuas}{%
\chapter{Patrones con marcas
continuas}\label{patrones-con-marcas-continuas}}

\hypertarget{correlaciuxf3n-de-marca}{%
\section{Correlación de marca}\label{correlaciuxf3n-de-marca}}

\hypertarget{variograma-de-marca}{%
\section{Variograma de marca}\label{variograma-de-marca}}

\hypertarget{modelos-nulos-no-paramuxe9tricos-randon-labeling-y-toroidal-shift}{%
\section{Modelos nulos no paramétricos: randon labeling y toroidal
shift}\label{modelos-nulos-no-paramuxe9tricos-randon-labeling-y-toroidal-shift}}

\hypertarget{modelos-nulos-paramuxe9tricos-no-espaciales-simulando-glms}{%
\section{modelos nulos paramétricos (no espaciales): simulando
GLMs}\label{modelos-nulos-paramuxe9tricos-no-espaciales-simulando-glms}}

\hypertarget{anuxe1lisis-de-patrones-con-marcas-funcionales}{%
\section{Análisis de patrones con marcas
funcionales}\label{anuxe1lisis-de-patrones-con-marcas-funcionales}}

\bookmarksetup{startatroot}

\hypertarget{patrones-con-marcas-discretas}{%
\chapter{Patrones con marcas
discretas}\label{patrones-con-marcas-discretas}}

\hypertarget{funciones-bivariadas-eg.-kcross-kdot}{%
\section{Funciones bivariadas (eg. Kcross,
Kdot)}\label{funciones-bivariadas-eg.-kcross-kdot}}

\hypertarget{funciones-trivariadas-de-la-cruz-et-al-2008}{%
\section{Funciones trivariadas: De la Cruz et al
2008)}\label{funciones-trivariadas-de-la-cruz-et-al-2008}}

\hypertarget{idar}{%
\section{Idar}\label{idar}}

\hypertarget{modelos-nulos-no-paramuxe9tricos-randon-labeling-y-toroidal-shift-1}{%
\section{Modelos nulos no paramétricos: randon labeling y toroidal
shift}\label{modelos-nulos-no-paramuxe9tricos-randon-labeling-y-toroidal-shift-1}}

\hypertarget{modelos-nulos-paramuxe9tricos-no-espaciales-simulando-glms-1}{%
\section{modelos nulos paramétricos (no espaciales): simulando
GLMs}\label{modelos-nulos-paramuxe9tricos-no-espaciales-simulando-glms-1}}

\bookmarksetup{startatroot}

\hypertarget{patrones-con-marcas-discretas-y-continuas}{%
\chapter{Patrones con marcas discretas y
continuas}\label{patrones-con-marcas-discretas-y-continuas}}

\hypertarget{funciones-cruzadas-eg.-mrkcross-muxeda-en-nixon}{%
\section{Funciones cruzadas (eg. mrkcross Mía en
Nixon)}\label{funciones-cruzadas-eg.-mrkcross-muxeda-en-nixon}}

\hypertarget{ifdar-y-similares}{%
\section{IFDAR y similares}\label{ifdar-y-similares}}

\hypertarget{modelos-nulos-no-parametricos-randon-labeling-y-toroidal-shift}{%
\section{Modelos nulos no parametricos: randon labeling y toroidal
shift}\label{modelos-nulos-no-parametricos-randon-labeling-y-toroidal-shift}}

\hypertarget{modelos-nulos-paramuxe9tricos-no-espaciales-simulando-glms-2}{%
\section{modelos nulos paramétricos (no espaciales): simulando
GLMs}\label{modelos-nulos-paramuxe9tricos-no-espaciales-simulando-glms-2}}

\bookmarksetup{startatroot}

\hypertarget{patrones-de-puntos-masivos}{%
\chapter{Patrones de ``puntos''
masivos}\label{patrones-de-puntos-masivos}}

\hypertarget{pescador-et-al.-halo}{%
\section{Pescador et al.~``Halo''}\label{pescador-et-al.-halo}}

\hypertarget{pescador-et-al.-overlapp-test}{%
\section{Pescador et al.~Overlapp
test}\label{pescador-et-al.-overlapp-test}}

\hypertarget{wiegand-et-al.-2006}{%
\section{Wiegand et al.~2006}\label{wiegand-et-al.-2006}}

\bookmarksetup{startatroot}

\hypertarget{patrones-tridimensionales}{%
\chapter{Patrones tridimensionales}\label{patrones-tridimensionales}}

\hypertarget{ejemplo-orquuxeddeas-luxedquenes-isable-tec.}{%
\section{Ejemplo orquídeas, líquenes isable,
tec.}\label{ejemplo-orquuxeddeas-luxedquenes-isable-tec.}}

\bookmarksetup{startatroot}

\hypertarget{patrones-espacio-temporales}{%
\chapter{Patrones
espacio-temporales}\label{patrones-espacio-temporales}}

\hypertarget{xxxx.}{%
\section{XXXX.}\label{xxxx.}}

\bookmarksetup{startatroot}

\hypertarget{patrones-de-puntos-en-una-red-lineal}{%
\chapter{Patrones de puntos en una red
lineal}\label{patrones-de-puntos-en-una-red-lineal}}

\hypertarget{ejemplo-atropellos}{%
\section{ejemplo atropellos}\label{ejemplo-atropellos}}

\bookmarksetup{startatroot}

\hypertarget{referencias}{%
\chapter*{Referencias}\label{referencias}}
\addcontentsline{toc}{chapter}{Referencias}

\hypertarget{refs}{}
\begin{CSLReferences}{1}{0}
\leavevmode\vadjust pre{\hypertarget{ref-knuth84}{}}%
Knuth, Donald E. 1984. {``Literate Programming.''} \emph{Comput. J.} 27
(2): 97--111. \url{https://doi.org/10.1093/comjnl/27.2.97}.

\end{CSLReferences}



\end{document}
